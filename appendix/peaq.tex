\chapter{PEAQ Implementierungen}
\label{ch:peaq}

\section{EAQUAL}

\section{PQevalAudio}

\section{peaqb}

Bei \texttt{peaqb} handelt es sich um die Bemühung eine Implementation der Recommendation ITU-R BS.1387-1 als freie offene Software zu schaffen. Initiier im Jahr 2003 liegt seit dem 14. März 2013 das GPLv2 lizensierte Tool in der Version 1.0 Beta vor. Zu finden ist das Projekt auf der Softwareentwicklungsplattform Sourceforge\footnote{\url{http://sourceforge.net/projects/peaqb/}}, welche sich nicht ganz unbegr\"undet in den letzten Jahre den Beinahmen \glqq Friedhof der Open Source Projekte\grqq{} eingefangen hat. Der Grund daf\"ur ist auch an diesem Projekt leider zu sp\"uhren. 

Die Versuche \texttt{peaqb} für die Signal-Qualitätssicherung zu bem\"uhen haben schnell gezeigt, dass das Tool noch sehr instabil ist. In den meisten F/"allen terminiert das Programm mit einem Segmentation Fault. Falls es in der Lage ist Endergebnisse zu berechnen, so sind diese oftmals ähnlich zu PQevalAudio unbrauchbar, da sie entweder ebenfalls in nicht definierten Bereichen liegen oder mit der subjektiven H\"orwahrnehmung einfach nicht \"ubereinstimmen. 

Eine Weiterentwicklung jenseits einer Version 1.0 Beta ist nicht abzusehen. 

\section{OPERA}





