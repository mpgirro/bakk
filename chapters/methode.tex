%%%%%%%%%%%%%%%%%%%%%%%%%%%%%%%%%%%%%%%%%
%%%    METHODE    %%%%%%%%%%%%%%%%%%%%%%%
%%%%%%%%%%%%%%%%%%%%%%%%%%%%%%%%%%%%%%%%%

\section{Framework}

Modell der embedding und extraction phase

\section{Watermark Implantierung}

\subsection{Synchronisations-Codes}

wozu sync codes ganz allgemein

\subsection{Fehlerkorrekturverfahren}

Error detection and correction

\subsubsection{BCH-Codes}

\subsubsection{RS-Codes}

\subsubsection{Turbo-Codes}

\subsection{Datenstrukturen und Protokoll}

\subsection{Qualitätskontrolle mittels }

\subsubsection{Signal-Rauschabstand}

\subsubsection{Objective Difference Grade}

\paragraph{PQevalAudio}

\paragraph{EAQUAL}

\paragraph{peaqb}


\section{Watermark Extrahierung}


\subsection{Resynchronisaton und Interpolation}

\subsection{Synchronisations-Code Erkennung}

\subsubsection{Barker-Codes}

vorteil der barker codes, autocorrelation

\subsection{Datenextrahierung}



