\chapter{Stirmark for Audio Testautomation}
\label{ch:stirmarkaudio}

Das Verlangen der \textit{Stirmark for Audio} API nach nicht näher definierten Streams kann problematisch sein. Exemplarisch sei hier deswegen ein kurzes Bash-Skript gebracht, das dessen Verwendung illustrieren soll:
      
\lstset{
	escapeinside={\%*}{*)},
}  
\begin{lstlisting}[language=bash,
				   caption=Anwendungsbeispiel für Stirmark for Audio,
				   captionpos=b]
#!/usr/bin/env bash

streamfile="teststream"
tooldir="../bin" # smfa2 and read_write_stream binary location

if [ $# -ne 3 ] ; then
   echo "usage: ./stirmark_test <testfile> <samplerate> <channels>"
   exit 1
fi

testfile=$1
samplerate=$2
channels=$3

attack_list=( "AddDynNoise:20", "AddSinus:900 1300", "Invert" )
			
# first convert the audio file to a stream
$tooldir/read_write_stream -f $testfile -c $channels > $streamfile
 
# run all attacks and convert results to audiofiles
for attack in "${attack_list[@]}" ; do
   attack_name=${attack%%:*}
   attack_param=${attack#*:}
   $tooldir/smfa2 --$attack_name $attack_param -s $samplerate < $streamfile \
     | $tooldir/read_write_stream -p -s $samplerate -c $channels -f \
     ${testfile%.wav}-attacked-$attack_name-${attack_param// /-}.wav
done
\end{lstlisting}









